\documentclass[11pt]{beamer}
\usetheme{Warsaw}
\usepackage[utf8]{inputenc}
\usepackage[portuguese]{babel}
\usepackage[T1]{fontenc}
\usepackage{amsmath}
\usepackage{amsfonts}
\usepackage{amssymb}
\usepackage{graphicx}

% Colocando numero de paginas no slide
\setbeamertemplate{footline}[frame number]

% Desativando os botoes de navegacao
\beamertemplatenavigationsymbolsempty

\author{Augusto Cesar de Aquino Ribas}

\title[Bioestatística]{Bioestatística - Aula 1 \\
Introdução ao curso e a linguagem R}
%\setbeamercovered{transparent} 
%\setbeamertemplate{navigation symbols}{} 
%\logo{} 
\institute{Teste} 
\date{} 
%\subject{} 
\begin{document}



\begin{frame}[plain]
\titlepage
\end{frame}

\begin{frame}
\tableofcontents
\end{frame}

\section{Apresentação}

\begin{frame}{nome frame}

\begin{block}{Exemplo}
Este é um ambiente chamado \emph{block} com um titulo \emph{Exemplo}.
\end{block}

\end{frame}

\section{Linguagem R}
\begin{frame}

teste

\end{frame}


\begin{frame}{Oi}

\end{frame}

\end{document}